%%%23:32:38 13/10/2015 -VieTeX creates C:\Studys\Maths\CacMoHinhNgauNhien\Markov\Markov\KetLuan.tex
%Đại học bách khoa Hà Nội
%Viện toán ứng dụng và tin học
%Lớp 					: 	Toán Tin K56.
%---------------------------------------
%Kỳ học				:  2015 - 2016
%Môn học			: Các mô hình ngẫu nhiên
%Giảng viên		: TS.Nguyễn Thị Ngọc Anh
%_______________________________________
%Tên nhóm			:	Nhóm 3
%Nhóm trưởng	: 	Nguyễn Thế Thức
%Thành viên		: 	Vũ Thành Đạt
%								Đặng Văn Tòng
%								Nguyễn Văn Kiên
%_______________________________________
%-------------------------------------------------------------------------------------------------------------------------------------------------------------------------------------
%+ Kết luận: Trình bày kết quả đã làm được và các hướng phát triển trong tương lai
%+-----------------------------------------------------------------------------------------------------------------------------------------------------------------------------------
\centerline{\bf \large\MakeUppercase{Kết luận}}
\vspace{20pt}
\normalsize{
Các công việc nhóm 1 đã làm được
\begin{itemize}
\item[1] Hiểu, nắm bắt các cơ sở lý thuyết đã học. Ứng dụng trong việc xây dựng giải thuật để giải quyết các bài toán đặt ra.
\item[2] Xây dựng chương trình mô phỏng các bài toán. 
\end{itemize}
Tuy nhiên báo cáo còn nhiều sai sót, hạn chế, nhóm em rất mong nhận được sự góp ý của quý thầy cô và các bạn.
}


